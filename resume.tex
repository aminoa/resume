\documentclass{resume} % Use the custom resume.cls style

\usepackage[left=0.5 in,top=0.4in,right=0.5 in,bottom=0.4in]{geometry} % Document margins
\usepackage{hyperref}
\newcommand{\tab}[1]{\hspace{.2667\textwidth}\rlap{#1}} 
\newcommand{\itab}[1]{\hspace{0em}\rlap{#1}}
\name{Aneesh Maganti}

%Headers
\address { 
    \href{mailto:aneesh.maganti@nyu.edu}{aneesh.maganti@nyu.edu} • 
    \href{github.com/aminoa}{github.com/aminoa} •
    \href{linkedin.com/in/aneesh-maganti}{linkedin.com/in/aneesh-maganti}
}

\begin{document}

%----------------------------------------------------------------------------------------
% Education
%----------------------------------------------------------------------------------------

\begin{rSection}{Education}
    \textbf{New York University}, Tandon School of Engineering, Brooklyn, NY \hfill {Fall 2024}\\
    Bachelor of Science, Computer Science \hfill GPA: \textbf {3.80} \\
    {\emph {Relevant Courses:}} {Algorithmic Machine Learning, Offensive Security, ML Visualization, Computer Architecture}
\end{rSection}

%----------------------------------------------------------------------------------------
% Skills
%----------------------------------------------------------------------------------------

\begin{rSection}{SKILLS}
    \begin{tabular}{ @{} >{\bfseries}l @{\hspace{8ex}} l }
        Languages & C++, Python, Javascript, C\#, Java, Bash \\
        Technologies & PyTorch, Next.js, SDL, QT, React, Node, PostgreSQL, Sklearn, Docker, Linux \\
    \end{tabular}
\end{rSection}
\smallskip
%----------------------------------------------------------------------------------------
% Experience
%----------------------------------------------------------------------------------------

\begin{rSection}{EXPERIENCE}

\textbf{Affirm}, New York City, NY, {\emph{Software Engineer, Infrastructure}} \hfill July 2025 - Present

\textbf{Qualcomm}, San Diego, CA, {\emph{Software Engineer Intern, AI Software/ML Group}} \hfill May 2024 - Aug 2024 \\ 
• Implemented various memory optimizations across the Qualcomm AI Runtime (QAIRT), reducing peak memory usage by 20\% across benchmarked ResNet50 and LLAMA2 and LLAMA3 LLM models \\
• Optimized AIMET quantization pipeline via C++ PyBind API to reference static graph tensors within Python subprocess, reducing in-memory allocations by 50\% \\
• Refactored QAIRT/AIMET toolchain via PyTorch 'meta' device to prevent unnecessary weight initialization and modified model export to use in-memory IR graph, reducing disk space and improving runtime \smallskip

\textbf{New York University}, Brooklyn, NY, {\emph{Teaching Assistant (Machine Learning)}} \hfill Sep 2023 - Dec 2023 \\
• Instructed students weekly for machine learning topics of written and programming tasks through office hours \\
• Graded weekly assignments on the basis of proper algorithm implementation, code correctness and style \smallskip

\textbf {NYU Algorithms and Foundations Group}, Brooklyn, NY, {\emph{ML Academic Researcher}} \hfill Feb 2023 - Sep 2023 \\
• Designed and implemented \textit{Deltagonalshift}, a diagonal estimator for a dynamic matrix under Professor Christopher Musco to improve neural network optimization based on DeltaShift trace estimation algorithm \newline
• Demonstrated Deltagonalshift was more effective than repeatedly running Hutchinson's diagonal estimator \smallskip

\textbf{MarketFusion}, Los Altos, CA, {\emph{Software Engineering Intern}} \hfill July 2022 - Sep 2022 \\
• Developed client-side React.js web application registration and shopping pages for online food delivery service \\
• Facilitated user account creation by sending server requests to internal MySQL database via the Axios library \smallskip
% • Revamped login authentication by client-side via regular expressions and server-side to aid the website security 

\textbf{Corelink}, Brooklyn, NY, {\emph{Software Engineer Intern}}\ \hfill Sep 2021 - May 2022 \\
• Implemented a C++ UDP network packet splitter to enable researchers to bypass Corelink's MTU limit from 20,000 to 64,000 bytes, increasing maximum throughput by 220\% \newline
• Designed Next.js/React interview scheduling platform using Auth0 for authentication and MongoDB backend 

\end{rSection} 

%----------------------------------------------------------------------------------------
% Projects
%----------------------------------------------------------------------------------------

\begin{rSection}{PROJECTS/ACTIVITIES}

\textbf{\href{https://bugs-nyu.github.io/}{BUGS Open Source Club President}} \hfill Sep 2022 - Dec 2023 \\
• Started and led 50+ member club by coordinating biweekly workshop and project coding events to discuss software engineering skills, foster contributions to open source, and create a fun, inclusive CS community.\\
• Led multiple workshops including discussion of open-source licenses, JavaScript Playwright automation, and an overview of emulation and the internals of C++ Game Boy emulator.\\
• Developed Next.js-React website NYU Syllabi with Netlify hosting and Docusaurus-based NYU CS Wiki websites

\textbf{\href{https://github.com/aminoa/dot-matrix}{Dot Matrix - Game Boy Emulator}} \hfill August 2023 \\
• Designed x86 C++ emulator for the Game Boy platform by implementing 255 standard + 240 cb instructions \\
• Simulated hardware features including registers, graphics (SDL), memory, timers, interrupts, and input handling 

\textbf {\href{https://github.com/aminoa/sentitweet/}{SentiTweet}} \hfill March 2023 - April 2023 \\
• Created sentimental tweet generator via modified PyTorch PPLM library with GPT-2 to simulate conversations between Twitter users using natural language generation\\
• Employed D3.js visualization library to create graph of tweets, their sentiments, and relationships 

\end{rSection}

\end{document}
