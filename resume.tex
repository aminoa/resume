\documentclass{resume} % Use the custom resume.cls style

\usepackage[left=0.5 in,top=0.4in,right=0.5 in,bottom=0.4in]{geometry} % Document margins
\usepackage{hyperref}
\newcommand{\tab}[1]{\hspace{.2667\textwidth}\rlap{#1}} 
\newcommand{\itab}[1]{\hspace{0em}\rlap{#1}}
\name{Aneesh Maganti}

%Headers
\address { 
    312-841-0636 •
    % \href{mailto:aneesh.maganti@nyu.com}{aneesh.maganti@nyu.edu} • 
    \href{mailto:asmaganti@gmail.com}{asmaganti@gmail.com} • 
    \href{github.com/aminoa}{github.com/aminoa} •
    \href{linkedin.com/in/aneesh-maganti}{linkedin.com/in/aneesh-maganti}
}

\begin{document}

%----------------------------------------------------------------------------------------
% Education
%----------------------------------------------------------------------------------------

\begin{rSection}{Education}
    \textbf{New York University}, Tandon School of Engineering, Brooklyn, NY \hfill {Jan 2025}\\
    Bachelor of Science, Computer Science \hfill GPA: \textbf {3.77} \\
    {\emph {Relevant Courses:}} {Offensive Security, Computer Architecture, Visualization in ML, Algorithmic Machine Learning}
\end{rSection}

%----------------------------------------------------------------------------------------
% Skills
%----------------------------------------------------------------------------------------

\begin{rSection}{SKILLS}
    \begin{tabular}{ @{} >{\bfseries}l @{\hspace{8ex}} l }
        Languages & C++, Python, Javascript, C\#, Java, Bash, \\
        Technologies & Next.js, SDL, QT, React, Node, PostgreSQL, Sklearn, Docker, Linux, PyTorch \\
    \end{tabular}
\end{rSection}
\smallskip
%----------------------------------------------------------------------------------------
% Experience
%----------------------------------------------------------------------------------------

\begin{rSection}{EXPERIENCE}
\textbf{New York University}, Brooklyn, NY, {\emph{Teaching Assistant (Machine Learning)}} \hfill Sep 2023 - Dec 2023 \\
• Instructed students weekly for machine learning topics of written and programming tasks through office hours \\
• Graded weekly assignments on the basis of proper algorithm implementation, code correctness and style 

\textbf {NYU Algorithms and Foundations Group}, Brooklyn, NY, {\emph{Researcher}} \hfill Feb 2023 - Sep 2023 \\
• Designed and implemented a diagonal estimator for a dynamic matrix, Deltagonalshift, based on Hutchinson's diagonal estimator and the DeltaShift trace estimation algorithm under Professor Christopher Musco \newline
• Demonstrated Deltagonalshift was more effective than repeatedly running Hutchinson's diagonal estimator. 
% • Developed AdaHessian example via PyTorch to test Deltagonalshift compared to its current diagonal esimator

% \textbf{MarketFusion}, Remote, {\emph{Software Engineering Intern}} \hfill July 2022 - Sep 2022 \\
% • Developed client-side React.js web application registration and shopping pages for online food delivery service \\
% • Facilitated user account creation by sending server requests to internal MySQL database via the Axios library \\
% • Revamped login authentication by implementing 4 unique character password checks and a length requirement of 8-20 characters client-side via regular expressions and server-side to aid the security of the website 

\textbf{Corelink}, Brooklyn, NY, {\emph{Software Engineering Intern}}\ \hfill Sep 2021 - May 2022 \\
• Implemented a C++ UDP network packet splitter to enable researchers to bypass Corelink's MTU limit from 20,000 to 64,000 bytes, increasing maximum throughput by 220\% \newline
• Designed Next.js/React interview scheduling platform using Auth0 for authentication and MongoDB backend \newline
% • Guided 3 students as a part of my team, assisting them with their project design and implementation \newline
• Scripted bash memory tests to determine the effectiveness of RDMA (Remote Direct Memory Access) 

\textbf{Monarc}, Dallas, TX, {\emph{Software Engineering Intern}} \hfill Jun 2021 - Aug 2021 \\
• Developed C\# UWP desktop application pages using MVVM (Model-View View-Model) principles to manipulate a robotic football quarterback to throw balls at 5 placements and distances up to 100 yards \newline
• Devised error checks and boot logging to enable remote debugging, improving the stability of the machine \smallskip
\end{rSection} 


%----------------------------------------------------------------------------------------
% Projects
%----------------------------------------------------------------------------------------

\begin{rSection}{PROJECTS/ACTIVITIES}

\textbf{\href{https://bugs-nyu.github.io/}{BUGS Open Source Club President}} \hfill Sep 2022 - Dec 2023 \\
• Started and currently lead over 50 member club by coordinating biweekly workshop and project coding events to discuss software engineering skills, foster contributions to open source, and create a fun, inclusive CS community.\\
• Led multiple workshops including discussion of open-source licenses, JavaScript Playwright automation, and an overview of emulation and the internals of C++ Game Boy emulator.\\
• Developed Next.js-React website NYU Syllabi with Netlify hosting and Docusaurus-based NYU CS Wiki websites

% • Developed FOSS projects Gallatin (course scheduler), NYU Syllabi, NYU CS Wiki with member contributions
% \textbf{\href{https://github.com/osirislab/CSAW-CTF-2023-Finals/tree/main/rev/cell}{Cell - CSAW 2023 Finals Reverse Engineering CTF}} \hfill November 2023 \\
% • Designed C++ capture the flag reverse engineering challenge for the Playstation 3 via docker PSL1GHT SDK \\
% • Solution required Ghidra decompilation or usage of fail0verflow tools to determine exact controller inputs \\
% • Released at start of NYU CSAW 2023 and solved by 5 out of 50 teams globally

\textbf{\href{https://github.com/aminoa/dot-matrix}{Dot Matrix - Game Boy Emulator}} \hfill August 2023 \\
• Designed x86 C++ emulator for the Game Boy platform by implementing 255 standard + 240 cb instructions \\
• Simulated hardware features including registers, graphics (SDL), memory, timers, interrupts, and input handling \newline
• Implementing Pixel Processing Unit timing methods and drawing functionality using SDL graphics library

\textbf {\href{https://github.com/aminoa/sentitweet/}{SentiTweet}} \hfill March 2023 - April 2023 \\
• Created sentimental tweet generator via modified PyTorch PPLM library with GPT-2 to simulate conversations between Twitter users using natural language generation\\
• Employed D3.js visualization library to create graph of tweets, their sentiments, and relationships 

\textbf{\href{https://github.com/aminoa/chip8}{Chip8 Emulator}} \hfill August 2022 \\
• Designed x86 C++ interpreter for the Chip8 platform by emulating its 35 standard opcodes and its specifications
• Wrote separate disassembler to obtain assembly instructions from program binary data

\end{rSection}

\end{document}