\documentclass{resume} % Use the custom resume.cls style

\usepackage[left=0.4 in,top=0.4in,right=0.4 in,bottom=0.4in]{geometry} % Document margins
\newcommand{\tab}[1]{\hspace{.2667\textwidth}\rlap{#1}} 
\newcommand{\itab}[1]{\hspace{0em}\rlap{#1}}
\name{Aneesh Maganti} % Your name

%This resume being made right now was from 12-26-2021, will change it but first translating it to a new format
%Header
\address { 
    (312) 841-0636 • 
    Brooklyn, NY • 
    \href{mailto:aneesh.maganti@nyu.com}{aneesh.maganti@nyu.edu} • 
    \href{github.com/aminoa}{github.com/aminoa}
}

\begin{document}

%----------------------------------------------------------------------------------------
% Education
%----------------------------------------------------------------------------------------

\begin{rSection}{Education}
    {\bf New York University}, Tandon School of Engineering, Brooklyn, NY \hfill {May 2024}\\
    Bachelor of Science, Computer Science \hfill GPA: {\bf 3.7} \\
    {\emph {Relevant Courses:}} {Machine Learning, Algorithms, Operating Systems, OOP, Data Structures}
\end{rSection}

%----------------------------------------------------------------------------------------
% Skills
%----------------------------------------------------------------------------------------

\begin{rSection}{SKILLS}
    \begin{tabular}{ @{} >{\bfseries}l @{\hspace{8ex}} l }
        Languages & C++, Typescript, Python, C\#, Java \\
        Frameworks/Libraries & React, Django, Node.js, Qt, SQL \\
        Operating Systems & Windows, Unix \\
    \end{tabular}
\end{rSection}
\smallskip
%----------------------------------------------------------------------------------------
% Experience
%----------------------------------------------------------------------------------------

%Current doesn't have the proper spacing after the bullets but will fix later
\begin{rSection}{EXPERIENCE}

\textbf{\bf Monarc}, Dallas, TX, {\emph{Software Engineering Intern}} \hfill Aug 2021 - Present\\
• Devised error checks and boot logging to improve Seeker performance and enable remote debugging \newline
• Applied MVVM principles to develop new UI/UX features to enhance the Seeker machine feature-set \smallskip

\textbf{\bf Software Developer (Researcher?) at Corelink High-Speed Research Network} \hfill Sep 2021 - Present \\%font is too big here
• Helped other teammates with their projects 
• Researched Corelink’s network architecture and RDMA/InfiniBand protocol to identify novel applications in Corelink infrastructure \newline
• Ran memory tests to determine whether using RDMA would enable researchers to dramatically increase their current processes and analysis of data on the Corelink research network
- Also wrote powershell downloader for different repositor
- Current work on network splitting 
- Managing three other VIP students

\end{rSection} 
%----------------------------------------------------------------------------------------
% Projects
%----------------------------------------------------------------------------------------
\begin{rSection}{PROJECTS}

\textbf{\bf bkRoad } \hfill March 2022 \\
- Participated in the Amazon Lightsail Container Hackathon \newline
- Wrote Next.js application and handled connection to SQL Amazon DynamoDB \newline
- Hosted on Amazon Lightsail \newline
- Won 2nd place in the hackathon \newline

\textbf{\bf Interview Automation - HackNYU} \hfill February 2022 \newline
- Student/Admin logon page, teams list, make admin/questions, applicant information (nanme, netid, status) \newline
- Wrote Next.js react pages and linked them to SQL db via Mongoose \newline
- Configuration of NYU SSO logon via Auth0 \newline
- Connected to gmail api to send email \newline



\textbf{\bf Auto Daily Screener } \hfill Sep 2021 - Present \\%font is too big here
• Built command line app to automate the NYU Daily Screener via Selenium web automation
\newline
• Designed a QT GUI for the application and used PyInstaller to build for Windows

\textbf{\bf Concussion Detection Device Project }, Tandon School of Engineering \hfill Sep 2020 - Dec 2020 \\%font is too big here
• Built Arduino-based IoT device to detect concussions for athletes using a pressure sensor
\newline
• Developed code to detect pressure levels, change device LED color, and send email to parents/guardians notifying them of potential injury to athlete

% \textbf{\bf Tandon Syllabi } \hfill October 2020 - Present \\%font is too big here
% • Wrote online repository for syllabi and other class materials for the NYU Tandon School of Engineering, using React for the frontend
% \newline
% • Currently has 8 syllabi hosted with the ability for students to add their own 

\textbf{\bf Implemented Shortcut Feature in Dolphin } \hfill October 2020 - Present \\%font is too big here
• Developed and implemented highly desired ‘Add Shortcut to Desktop’ feature to the open-source emulator Dolphin via the native Windows API, the Qt library, and C++
\newline
• Collaborated with other open-source contributors to refactor code and add it to Dolphin Emulator


\end{rSection} 

\begin{rSection}{Extra-Curricular Activities} 
\begin{itemize} %left this because there was space to do this
    \item 	Poly Programming Team
    \item   Society of Asian Scientists and Engineers

\end{itemize}


\end{rSection}

\end{document}
