\documentclass{resume} % Use the custom resume.cls style

\usepackage[left=0.4 in,top=0.4in,right=0.4 in,bottom=0.4in]{geometry} % Document margins
\newcommand{\tab}[1]{\hspace{.2667\textwidth}\rlap{#1}} 
\newcommand{\itab}[1]{\hspace{0em}\rlap{#1}}
\name{Aneesh Maganti} % Your name

%Header
\address { 
    (312) 841-0636 • 
    New York City, NY • 
    \href{mailto:aneesh.maganti@nyu.com}{aneesh.maganti@nyu.edu} • 
    \href{github.com/aminoa}{github.com/aminoa}
}

\begin{document}

%----------------------------------------------------------------------------------------
% Education
%----------------------------------------------------------------------------------------

\begin{rSection}{Education}
    {\bf New York University}, Tandon School of Engineering, Brooklyn, NY \hfill {May 2024}\\
    Bachelor of Science, Computer Science \hfill GPA: {\bf 3.61} \\
    {\emph {Relevant Courses:}} {Machine Learning, Algorithms, Operating Systems, OOP, Data Structures}
\end{rSection}

%----------------------------------------------------------------------------------------
% Skills
%----------------------------------------------------------------------------------------

\begin{rSection}{SKILLS}
    \begin{tabular}{ @{} >{\bfseries}l @{\hspace{8ex}} l }
        Languages & C++, Typescript, Python, Go, C\#, Java \\
        Frameworks/Libraries & React, Node.js, Qt, SQL \\
        Operating Systems & Windows, Unix \\
    \end{tabular}
\end{rSection}
\smallskip
%----------------------------------------------------------------------------------------
% Experience
%----------------------------------------------------------------------------------------

%Current doesn't have the proper spacing after the bullets but will fix later
\begin{rSection}{EXPERIENCE}

\textbf{\bf Monarc}, Dallas, TX, {\emph{Software Engineering Intern}} \hfill Aug 2021 - Jan 2022\\
• Devised error checks and boot logging to improve Seeker performance and enable remote debugging \newline
• Applied MVVM principles to develop new UI/UX features to enhance the Seeker machine feature-set \smallskip

\textbf{\bf Corelink High-Speed Research Network}, {\emph{Academic Researcher}} \ \hfill Sep 2021 - Present \\%maybe designed and implemented? (didn't do that because it wnould take another line)
• Researched the Corelink network infrastructure and implemented a network packet splitter for UDP connections \newline
• Coordinated VIP students and assisting them with the management of their projects \newline
• Researched Corelink’s network architecture and RDMA/InfiniBand protocol and ran memory tests to determine the protocol's effectiveness for NYU researchers

\end{rSection} 
%----------------------------------------------------------------------------------------
% Projects
%----------------------------------------------------------------------------------------
\begin{rSection}{PROJECTS}

\textbf{\bf NYU Syllabi } \hfill June 2022 \\
• Wrote Next.js web application to provide easy access to syllabi across different fields at NYU \newline
• Connected to AWS RDS PostgreSQL database, hosted on AWS S3 storage, and dockerized for deployment

\textbf{\bf bkRoad - Amazon Lightsail Containers Hackathon } \hfill March 2022 \\
• Won 2nd place in the hackathon \newline
• Wrote Next.js application that allowed users to discover books, learn details about them, and loan them. \newline
• Handled connections to SQL Amazon DynamoDB and hosted on Amazon Lightsail

\textbf{\bf Interview Automation - HackNYU} \hfill February 2022 \newline
• A Next.js application to assist with interviewing candidates for the NYU Corelink team by providing scheduling, quizzing and management services\newline
• Wrote React pages for the question page, admin creation page, and applicant information \newline
• Handled authentication via Auth0 configuration of NYU SSO logon  \newline
• Connected to Gmail API to send emails on confirmation

\textbf{\bf Auto Daily Screener } \hfill Sep 2021 \\
• Built command line app to automate the NYU Daily Screener via Selenium web automation
\newline
• Designed QT GUI for the application and used PyInstaller to build for Windows \newline
• Highlighted in NYU Washington Square News

% \textbf{\bf Concussion Detection Device Project }, Tandon School of Engineering \hfill Sep 2020 - Dec 2020 \\
% • Built Arduino-based IoT device to detect concussions for athletes using a pressure sensor
% \newline
% • Developed code to detect pressure levels, change device LED color, and send email to parents/guardians notifying them of potential injury to athlete

% \textbf{\bf Tandon Syllabi } \hfill October 2020 - Present \\%font is too big here
% • Wrote online repository for syllabi and other class materials for the NYU Tandon School of Engineering, using React for the frontend
% \newline
% • Currently has 8 syllabi hosted with the ability for students to add their own 

\textbf{\bf Implemented Shortcut Feature in Dolphin } \hfill Oct 2020  \\%font is too big here
• Developed and implemented highly desired ‘Add Shortcut to Desktop’ feature to the open-source emulator Dolphin via the native Windows API, the Qt library, and C++
\newline
• Collaborated with other open-source contributors to refactor code and add it to Dolphin Emulator


\end{rSection} 

\begin{rSection}{Extra-Curricular Activities} 
\begin{itemize} %left this because there was space to do this
    \item 	Poly Programming Team
    \item   Society of Asian Scientists and Engineers

\end{itemize}


\end{rSection}

\end{document}
